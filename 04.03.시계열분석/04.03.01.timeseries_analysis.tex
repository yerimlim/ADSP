\documentclass[]{article}
\usepackage{lmodern}
\usepackage{amssymb,amsmath}
\usepackage{ifxetex,ifluatex}
\usepackage{fixltx2e} % provides \textsubscript
\ifnum 0\ifxetex 1\fi\ifluatex 1\fi=0 % if pdftex
  \usepackage[T1]{fontenc}
  \usepackage[utf8]{inputenc}
\else % if luatex or xelatex
  \ifxetex
    \usepackage{mathspec}
  \else
    \usepackage{fontspec}
  \fi
  \defaultfontfeatures{Ligatures=TeX,Scale=MatchLowercase}
\fi
% use upquote if available, for straight quotes in verbatim environments
\IfFileExists{upquote.sty}{\usepackage{upquote}}{}
% use microtype if available
\IfFileExists{microtype.sty}{%
\usepackage{microtype}
\UseMicrotypeSet[protrusion]{basicmath} % disable protrusion for tt fonts
}{}
\usepackage[margin=1in]{geometry}
\usepackage{hyperref}
\hypersetup{unicode=true,
            pdfborder={0 0 0},
            breaklinks=true}
\urlstyle{same}  % don't use monospace font for urls
\usepackage{color}
\usepackage{fancyvrb}
\newcommand{\VerbBar}{|}
\newcommand{\VERB}{\Verb[commandchars=\\\{\}]}
\DefineVerbatimEnvironment{Highlighting}{Verbatim}{commandchars=\\\{\}}
% Add ',fontsize=\small' for more characters per line
\usepackage{framed}
\definecolor{shadecolor}{RGB}{248,248,248}
\newenvironment{Shaded}{\begin{snugshade}}{\end{snugshade}}
\newcommand{\KeywordTok}[1]{\textcolor[rgb]{0.13,0.29,0.53}{\textbf{#1}}}
\newcommand{\DataTypeTok}[1]{\textcolor[rgb]{0.13,0.29,0.53}{#1}}
\newcommand{\DecValTok}[1]{\textcolor[rgb]{0.00,0.00,0.81}{#1}}
\newcommand{\BaseNTok}[1]{\textcolor[rgb]{0.00,0.00,0.81}{#1}}
\newcommand{\FloatTok}[1]{\textcolor[rgb]{0.00,0.00,0.81}{#1}}
\newcommand{\ConstantTok}[1]{\textcolor[rgb]{0.00,0.00,0.00}{#1}}
\newcommand{\CharTok}[1]{\textcolor[rgb]{0.31,0.60,0.02}{#1}}
\newcommand{\SpecialCharTok}[1]{\textcolor[rgb]{0.00,0.00,0.00}{#1}}
\newcommand{\StringTok}[1]{\textcolor[rgb]{0.31,0.60,0.02}{#1}}
\newcommand{\VerbatimStringTok}[1]{\textcolor[rgb]{0.31,0.60,0.02}{#1}}
\newcommand{\SpecialStringTok}[1]{\textcolor[rgb]{0.31,0.60,0.02}{#1}}
\newcommand{\ImportTok}[1]{#1}
\newcommand{\CommentTok}[1]{\textcolor[rgb]{0.56,0.35,0.01}{\textit{#1}}}
\newcommand{\DocumentationTok}[1]{\textcolor[rgb]{0.56,0.35,0.01}{\textbf{\textit{#1}}}}
\newcommand{\AnnotationTok}[1]{\textcolor[rgb]{0.56,0.35,0.01}{\textbf{\textit{#1}}}}
\newcommand{\CommentVarTok}[1]{\textcolor[rgb]{0.56,0.35,0.01}{\textbf{\textit{#1}}}}
\newcommand{\OtherTok}[1]{\textcolor[rgb]{0.56,0.35,0.01}{#1}}
\newcommand{\FunctionTok}[1]{\textcolor[rgb]{0.00,0.00,0.00}{#1}}
\newcommand{\VariableTok}[1]{\textcolor[rgb]{0.00,0.00,0.00}{#1}}
\newcommand{\ControlFlowTok}[1]{\textcolor[rgb]{0.13,0.29,0.53}{\textbf{#1}}}
\newcommand{\OperatorTok}[1]{\textcolor[rgb]{0.81,0.36,0.00}{\textbf{#1}}}
\newcommand{\BuiltInTok}[1]{#1}
\newcommand{\ExtensionTok}[1]{#1}
\newcommand{\PreprocessorTok}[1]{\textcolor[rgb]{0.56,0.35,0.01}{\textit{#1}}}
\newcommand{\AttributeTok}[1]{\textcolor[rgb]{0.77,0.63,0.00}{#1}}
\newcommand{\RegionMarkerTok}[1]{#1}
\newcommand{\InformationTok}[1]{\textcolor[rgb]{0.56,0.35,0.01}{\textbf{\textit{#1}}}}
\newcommand{\WarningTok}[1]{\textcolor[rgb]{0.56,0.35,0.01}{\textbf{\textit{#1}}}}
\newcommand{\AlertTok}[1]{\textcolor[rgb]{0.94,0.16,0.16}{#1}}
\newcommand{\ErrorTok}[1]{\textcolor[rgb]{0.64,0.00,0.00}{\textbf{#1}}}
\newcommand{\NormalTok}[1]{#1}
\usepackage{graphicx,grffile}
\makeatletter
\def\maxwidth{\ifdim\Gin@nat@width>\linewidth\linewidth\else\Gin@nat@width\fi}
\def\maxheight{\ifdim\Gin@nat@height>\textheight\textheight\else\Gin@nat@height\fi}
\makeatother
% Scale images if necessary, so that they will not overflow the page
% margins by default, and it is still possible to overwrite the defaults
% using explicit options in \includegraphics[width, height, ...]{}
\setkeys{Gin}{width=\maxwidth,height=\maxheight,keepaspectratio}
\IfFileExists{parskip.sty}{%
\usepackage{parskip}
}{% else
\setlength{\parindent}{0pt}
\setlength{\parskip}{6pt plus 2pt minus 1pt}
}
\setlength{\emergencystretch}{3em}  % prevent overfull lines
\providecommand{\tightlist}{%
  \setlength{\itemsep}{0pt}\setlength{\parskip}{0pt}}
\setcounter{secnumdepth}{0}
% Redefines (sub)paragraphs to behave more like sections
\ifx\paragraph\undefined\else
\let\oldparagraph\paragraph
\renewcommand{\paragraph}[1]{\oldparagraph{#1}\mbox{}}
\fi
\ifx\subparagraph\undefined\else
\let\oldsubparagraph\subparagraph
\renewcommand{\subparagraph}[1]{\oldsubparagraph{#1}\mbox{}}
\fi

%%% Use protect on footnotes to avoid problems with footnotes in titles
\let\rmarkdownfootnote\footnote%
\def\footnote{\protect\rmarkdownfootnote}

%%% Change title format to be more compact
\usepackage{titling}

% Create subtitle command for use in maketitle
\newcommand{\subtitle}[1]{
  \posttitle{
    \begin{center}\large#1\end{center}
    }
}

\setlength{\droptitle}{-2em}
  \title{}
  \pretitle{\vspace{\droptitle}}
  \posttitle{}
  \author{}
  \preauthor{}\postauthor{}
  \date{}
  \predate{}\postdate{}


\begin{document}

\section{---}\label{section}

\section{\texorpdfstring{\# title: ``Chapter4.3. Timeseries Analysis in
R''}{\# title: Chapter4.3. Timeseries Analysis in R}}\label{title-chapter4.3.-timeseries-analysis-in-r}

\section{\texorpdfstring{\#author: ``Yerim
Lim''}{\#author: Yerim Lim}}\label{author-yerim-lim}

\section{\texorpdfstring{\#date: ``Feb 8,
2018''}{\#date: Feb 8, 2018}}\label{date-feb-8-2018}

\section{output:}\label{output}

\section{html\_notebook:}\label{html_notebook}

\section{theme: sandstone}\label{theme-sandstone}

\section{toc: yes}\label{toc-yes}

\section{pdf\_document:}\label{pdf_document}

\section{latex\_engine: xelatex}\label{latex_engine-xelatex}

\section{mainfont: NanumGothic}\label{mainfont-nanumgothic}

\section{editor\_options:}\label{editor_options}

\section{chunk\_output\_type: inline}\label{chunk_output_type-inline}

\section{---}\label{section-1}

\begin{Shaded}
\begin{Highlighting}[]
\NormalTok{knitr}\OperatorTok{::}\NormalTok{opts_chunk}\OperatorTok{$}\KeywordTok{set}\NormalTok{(}\DataTypeTok{comment=}\StringTok{""}\NormalTok{)}
\end{Highlighting}
\end{Shaded}

library(TTR) library(forecast)

\section{reading data
------------------------------------------------------------}\label{reading-data}

kings \textless{}-
scan(`\url{https://robjhyndman.com/tsdldata/misc/kings.dat}', skip=3)
births \textless{}-
scan(``\url{http://robjhyndman.com/tsdldata/data/nybirths.dat}'')
souvenir \textless{}-
scan(``\url{http://robjhyndman.com/tsdldata/data/fancy.dat}'')

\section{basic timeseries analysis
-----------------------------------------------}\label{basic-timeseries-analysis}

head(births) head(kings) head(souvenir) kingstimeseries \textless{}-
ts(kings) birthstimeseries \textless{}- ts(births, frequency=12,
start=c(1946,1)) birthstimeseries2\textless{}- ts(births, frequency=12,
start=c(1946,7)) birthstimeseries3 \textless{}- ts(births, frequency=10,
start=c(1946,1)) birthstimeseries4 \textless{}- ts(births, frequency=6,
start=c(1946,1)) head(birthstimeseries) head(birthstimeseries2)
head(birthstimeseries3) head(birthstimeseries4)

plot.ts(birthstimeseries) plot.ts(birthstimeseries2)
plot.ts(birthstimeseries3) plot.ts(kingstimeseries)

souvenirtimeseries \textless{}- ts(souvenir, frequency= 12,
start=c(1987,1)) souvenirtimeseries

logsouvenierstimeseries \textless{}- log(souvenirtimeseries)
plot.ts(logsouvenierstimeseries)

\section{decompose non-seasonal data
---------------------------------------------}\label{decompose-non-seasonal-data}

kingstimeseries.SMA3 \textless{}- SMA(kingstimeseries, n=3)
plot.ts(kingstimeseries.SMA3) kingstimeseries.SMA8 \textless{}-
SMA(kingstimeseries, n=8) plot.ts(kingstimeseries.SMA8)

\section{Decompose Seasonal data------}\label{decompose-seasonal-data}

birthstimeseries.components \textless{}- decompose(birthstimeseries) \#
decompose(birthstimeseries, type=``additive'') \#
decompose(birthstimeseries, type=``multiplicative'')
birthstimeseries.components\(seasonal birthstimeseries.components\)trend
birthstimeseries.components\$type

plot(birthstimeseries.components)
plot.ts(birthstimeseries.components\(seasonal) plot.ts(birthstimeseries.components\)trend)

\section{Seosonal adjusting
------------------------------------------------------}\label{seosonal-adjusting}

birthstimeseries.seasonally.adjusted \textless{}- birthstimeseries -
birthstimeseries.components\$seasonal
plot(birthstimeseries.seasonally.adjusted, main=``seasonally adjusted'')
\#계절적인 요소가 제거된 시계열 모형

op \textless{}- par(no.readonly = TRUE) par(mfrow=c(1,3))

plot(birthstimeseries, main=``timeseries'')
plot(birthstimeseries.components\$seasonal, main=``seasomal'')
plot(birthstimeseries.seasonally.adjusted, main=``seasonally adjusted'')
\#계절적인 요소가 제거된 시계열 모형

par(mfrow=c(1,1))


\end{document}
